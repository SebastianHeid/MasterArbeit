\section{Diffusion Models and Flow Matching}
\label{sec:diffModel}
Diffusion models \cite{DiffusionModelIntroPaper, DDPM, DDIM, LDM, SGM, SDE} are probabilistic generative models  that have achieved remarkable success in image generation tasks \cite{Flux, HiDream}, demonstrating exceptional performance in terms of image quality and image diversity. While they require significant computational resources and have slower inference compared to some alternatives \cite{VAE, NormalizingFlow, GAN}, they have become the dominant generative modeling paradigm in the research community.
The main principles of diffusion models are reflected in two processes motivated by non-equilibrium thermodynamics, namely the forward process and the reverse process. In the forward process, the original images are gradually perturbed by systematically adding Gaussian noise until only pure noise is left. The reverse process describes the iterative noise removal by the diffusion model until a clean image is obtained. During image generation with a diffusion model, one starts from randomly sampled noise and applies the diffusion model iteratively, progressively denoising at each step. This repeated application gradually produces a clean, high-quality image. \\
Over time, several different formulations of diffusion models were proposed. However, there are three dominant formulations recognized in the literature \cite{SurveyDM}: Denoising diffusion probabilistic models (\gls{DDPM}s) \cite{DiffusionModelIntroPaper, DDPM}, score-based generative models (\gls{SGM}s) \cite{SGM} and stochastic differential equations (\gls{SDE}s) \cite{SDE}. \\

 Recently, a generalization of DMs was proposed, known as flow matching or Flow-based models \cite{RectifiedFlowMatching, FlowMatching}. Unlike classical diffusion, flow  matching models directly learn a vector field that defines the path between the noise distribution and the data distribution. By enforcing a straight path trajectory, the sampling process becomes computationally more efficient. This formulation was successfully applied in several new models like Flux-dev \cite{Flux} or Sana \cite{Sana}.\\
 In the following, the different formulations of diffusion models, DDPM, DDIM, SGM and SDE, are presented folowed by an introduction to the concept of flow matching.


\subsection{Denoising Diffusion Probabilistic Models}
Unless stated otherwise, the discussion and mathematical formulations of \gls{DDPM} presented in this chapter are based on the lecture notes by Inga Strümke and Helge Langseth \cite{LectureNotesDDPM}. \\
In \gls{DDPM}s \cite{DDPM} the forward and reverse process are modeled as Markov Chain (\gls{MC}), meaning that each step only depends on the immediate previous step. 
This key concept is illustrated in fig. \ref{GraphicalModelDDPM}. Let $\mathbf{x}_0,...\mathbf{x}_T$ be the states of the \gls{MC} with $\mathbf{x}_0$ representing the clean image and $\mathbf{x}_T$ representing pure noise, then one step of the reverse process is given by $p_\theta(\mathbf{x}_{t-1}|\mathbf{x}_t)$. This shows that the slightly denoised state $\mathbf{x}_{t-1}$ just depends on state $\mathbf{x}_t$ and $\theta$ denotes the parameters of the trained diffusion model being used to predict $\mathbf{x}_{t-1}$. The forward process, which progressively adds noise proceeds from right to left in  fig. \ref{GraphicalModelDDPM} and is given by $q(\mathbf{x}_t |\mathbf{x}_{t-1})$ where no learned parameters are required.

\begin{figure}[H]
    \centering
    \includegraphics[width=0.6\textwidth]{images/Background/MarkovChainDiffProcess.png}  % adjust filename and width
    \caption{The graphical model of DDPM from \cite{DDPM}.}
    \label{GraphicalModelDDPM}
\end{figure}

\subsubsection{Forward Process}
Formally, the forward process is defined as a \gls{MC} that progressively injects noise into the clean image $\mathbf{x}_0\sim p_{\text{data}}$ which is transformed such that for a large number of steps $T$ the final latent variable $\mathbf{x}_T$ approximately follows a new often simpler prior distribution $p_{\text{prior}}$. This boundary condition is necessary so that in the reverse process, which starts with a sample of $p_{\text{prior}}$, the model is able to transfer it back to the image domain $p_{\text{data}}$. The transformation is governed by a variance schedule $\{ \beta_t\}_{t=1}^T$, e.g., a linear \cite{DDPM}, a cosine \cite{ImprovedDDPM} or a learnable \cite{VDM} schedule, to ensure smooth interpolation between $p_{\text{prior}}$ and $p_{\text{data}}$. The schedule dictates the incremental increase of variance $\beta_t$.


Concretely, the next state $\mathbf{x}_t$ within the \gls{MC} is computed by 
\begin{equation}
    q(\mathbf{x}_t | \mathbf{x}_{t-1}) = K(\mathbf{x}_t; \mathbf{x}_{t-1}, \beta_t)
\end{equation}

with the Markov kernel $K$, which is chosen as a Gaussian Markov diffusion kernel  in DDPM leading to the explicit form 

%The joint probability of the entire sequence $\mathbf{x}_{1:T}$, given a sampled $\mathbf{x}_0$ from the data distribution, is obtained by 
%\begin{equation}
%    q \left(\mathbf{x}_{1:T}|\mathbf{x}_0 \right) =\prod_{t=1}^T q \left(\mathbf{x}_t | \mathbf{x}_{t-1} \right) \quad.
%\end{equation}
%As a Gaussian kernel is chosen the explicit form of $q \left(\mathbf{x}_t | \mathbf{x}_{t-1} \right)$ is given by 
\begin{equation}
    q \left(\mathbf{x}_t | \mathbf{x}_{t-1} \right) := \mathcal{N} \left( \mathbf{x}_t; \sqrt{1-\beta_t}\mathbf{x}_{t-1}, \beta_t \mathbf{I} \right)
\end{equation}



and the prior distribution $p_\text{prior}(\mathbf{x}) = \mathcal{N}(\mathbf{x}; \mathbf{0}, \mathbf{I})$.

% A property of this formulation is that there is a closed-form solution for every $\mathbf{x}_t$ in the \gls{MC}. In general, sampling from a Gaussian distribution with mean $\boldsymbol{\mu}$ and covariance $\mathbf{\Sigma}$ can be achieved by $\mathbf{x} = \boldsymbol{\mu} + \mathbf{A}\mathbf{z}$ with $\mathbf{z} \sim \mathcal{N}(\mathbf{0},\mathbf{I})$ and $\mathbf{\Sigma = \mathbf{A}\mathbf{A}}^T$. Defining $\alpha_t = 1 - \beta_t$ and setting $\mathbf{A} = \sqrt{1-\alpha_t} \mathbf{I}$ any $\mathbf{x}_t$ of the \gls{MC} can be computed by 

A key property of this Gaussian formulation is the existence of a closed-form expression for sampling any $\mathbf{x}_{t}$.  At every timestep $t$ the intermediate latent variable is computed by
\begin{align}
\mathbf{x}_t &= \sqrt{\alpha_t} \, \mathbf{x}_{t-1} + \sqrt{1 - \alpha_t} \, \mathbf{z}_t \\
             &= \sqrt{\alpha_t \alpha_{t-1}} \, \mathbf{x}_{t-2} + \sqrt{\alpha_t (1 - \alpha_{t-1})} \, \mathbf{z}_{t-1} + \sqrt{1 - \alpha_t} \, \mathbf{z}_t \\
             &= \sqrt{\alpha_t \alpha_{t-1}} \, \mathbf{x}_{t-2} + \sqrt{1 - \alpha_t \alpha_{t-1}} \, \mathbf{z}_{t-1:t} \\
             &= \dots \\
             &= \sqrt{\bar{\alpha}_t} \, \mathbf{x}_0 + \sqrt{1 - \bar{\alpha}_t} \, \mathbf{z}_{0:t} .
\end{align}
where $\bar{\alpha}_t = \prod_{s=1}^t \alpha_s$ which shows that $q(\mathbf{x}_t|\mathbf{x}_0)$ follows a normal distribution
\begin{equation}
     q(\mathbf{x}_t | \mathbf{x}_{0}) = \mathcal{N} \left( \mathbf{x_t};\sqrt{\bar{\alpha}_t} \mathbf{x_0}, (1- \bar{\alpha}_t)\mathbf{I}  \right)
\end{equation}
This underlines that during the forward process the mean of $\mathbf{x}_t$ is gradually moved towards zero and the variance increases towards one. Therefore, the closed-form solution for $\mathbf{x}_t$ is 

\begin{equation}
    \mathbf{x_t} =  \sqrt{\bar{\alpha}_t} \mathbf{x_0} + \sqrt{(1- \bar{\alpha}_t)} \boldsymbol{\epsilon}
    \label{x_t}
\end{equation}
with $\boldsymbol{\epsilon} \sim \mathcal{N}(\mathbf{0}, \mathbf{I})$. This possibility to directly sample every $\mathbf{x}_t$ is crucial for efficient training.


\subsubsection{Reverse Process } 
The reverse process recovers the clear image by iteratively removing noise. The authors of \cite{feller1949stochastic} showed that in the limit of infinitesimal small steps the reverse process has the same distributional form as the forward process, a Gaussian. Particularly, in the reverse process $q(\mathbf{x}_{t-1}| \mathbf{x}_t)$ is the quantity of interest. However, it is not tractable due to 
\begin{equation}
    q(\mathbf{x}_{t-1}| \mathbf{x}_t) = q(\mathbf{x}_{t}| \mathbf{x}_{t-1}) \frac{q(\mathbf{x}_{t-1})}{q(\mathbf{x}_t)}
\end{equation}
using Bayes law where $q(\mathbf{x}_{t-1})$ and $q(\mathbf{x}_t)$ are marginal distributions requiring the integration over the whole distribution $q(\mathbf{x}_0)$. Therefore, it is modeled by a neural network parameterized by $\theta$ that should learn the Gaussian distribution for each step. Although $q(\mathbf{x}_{t-1}| \mathbf{x}_t)$ is not tractable $q(\mathbf{x}_{t-1}| \mathbf{x}_t, \mathbf{x}_0)$ is which is leveraged during training.  The joint distribution learned by the model is given by 
\begin{equation}
    p_\theta(\mathbf{x}_{1:T} |\mathbf{x}_0) = p(\mathbf{x}_T) \prod_{t=1}^T p_\theta(\mathbf{x}_{t-1} | \mathbf{x}_t)
\end{equation}
where $p(\mathbf{x}_T) = $ is pure noise and $p_\theta(\mathbf{x}_{t-1}|\mathbf{x}_t)$ is a Gaussian with learned mean and covariance 
\begin{equation}
    p_\theta(\mathbf{x}_{t-1}|\mathbf{x}_t) = \mathcal{N}\left( \mathbf{x}_{t-1}; \boldsymbol{\mu}_\theta(\mathbf{x}_t,t), \boldsymbol{\Sigma}_\theta(\mathbf{x}_t,t)  \right)
\end{equation}
As seen in the forward process, the variance of the noise follows a predefined schedule which means that the neural network just has to learn the mean $\boldsymbol{\mu}_\theta(\mathbf{x}_t,t)$ and the variance is fixed by $\boldsymbol{\Sigma}_\theta(\mathbf{x}_t,t) = \sigma_t^2 \mathbf{I}$ \quad..

\subsubsection{Objective Function} Diffusion models belong to the class of probabilistic generative models. A natural choice for the objective of these models is the negative log-likelihood $- \log p_\theta(\mathbf{x}_0)$. As discussed in sec. \ref{VariationaAutoEncoder} the likelihood is not always tractable. Therefore, the \gls{ELBO} loss is used as learning objective. In the following the \gls{ELBO} loss is derived for diffusion models. This derivation is mainly based on \cite{pawlowski2024physics} and \cite{DDPM}. Starting with the expectation of the negative log-likelihood

% \begin{align}
% - \mathbb{E}_{q(x_0)} \left[ \log p_\theta(x_0) \right] 
% &= - \mathbb{E}_{q(x_0)} \left[ \log \left( \int dx_1 \dots dx_T\, p(x_T) \prod_{t=1}^T p_\theta(x_{t-1} \mid x_t) \right) \right]  \\
% &= - \mathbb{E}_{q(x_0)} \left[ \log \left( \int dx_1 \dots dx_T\, p(x_T) q(x_1, \dots, x_T \mid x_0) \notag \\
% &\quad \prod_{t=1}^T \frac{p_\theta(x_{t-1} \mid x_t)}{q(x_t \mid x_{t-1})} \right) \right] \\
% &= - \mathbb{E}_{q(x_0)} \left[ \log \mathbb{E}_{q(x_1,\dots,x_T \mid x_0)} \left[ p(x_T) \prod_{t=1}^T \frac{p_\theta(x_{t-1} \mid x_t)}{q(x_t \mid x_{t-1})} \right] \right] 
% \end{align}



\begin{align}
&- \mathbb{E}_{q(\mathbf{x}_0)} \left[ \log p_\theta(\mathbf{x}_0) \right] \\
&= - \mathbb{E}_{q(\mathbf{x}_0)} \left[ \log \left( \int d\mathbf{x}_1 \dots d\mathbf{x}_T\, p(\mathbf{x}_T) \prod_{t=1}^T p_\theta(\mathbf{x}_{t-1} \mid \mathbf{x}_t) \right) \right] \\
&= - \mathbb{E}_{q(\mathbf{x}_0)} \Bigg[ \log \Bigg( \int d\mathbf{x}_1 \dots d\mathbf{x}_T\, p(\mathbf{x}_T)\, q(\mathbf{x}_1, \dots, \mathbf{x}_T \mid \mathbf{x}_0) \prod_{t=1}^T \frac{p_\theta(\mathbf{x}_{t-1} \mid \mathbf{x}_t)}{q(\mathbf{x}_t \mid \mathbf{x}_{t-1})} \Bigg) \Bigg] \\
&= - \mathbb{E}_{q(\mathbf{x}_0)} \left[ \log \, \mathbb{E}_{q(\mathbf{x}_1,\dots,\mathbf{x}_T \mid \mathbf{x}_0)} \left[ p(\mathbf{x}_T) \prod_{t=1}^T \frac{p_\theta(\mathbf{x}_{t-1} \mid \mathbf{x}_t)}{q(\mathbf{x}_t \mid \mathbf{x}_{t-1})} \right] \right]
\end{align}







using Jensen's inequality 
% \begin{align}
%     - \mathbb{E}_{q(x_0)} \left[ \log p_\theta(x_0) \right] 
%     & \leq  - \mathbb{E}_{q(x_0)} \left[ \mathbb{E}_{q(x_1,\dots,x_T \mid x_0)} \left[ \log \left( p(x_T) \prod_{t=1}^T \frac{p_\theta(x_{t-1} \mid x_t)}{q(x_t \mid x_{t-1})} \right) \right] \right] \\
%     & =  - \mathbb{E}_{q(x_0,\dots,x_T)} \left[ \log \left( p(x_T) \prod_{t=1}^T \frac{p_\theta(x_{t-1} \mid x_t)}{q(x_t \mid x_{t-1})} \right) \right] \\
%     & =   \mathbb{E}_{q(x_0,\dots,x_T)} \left[ -\log  p(x_T) - \sum_{t=1}^T  \log  \frac{p_\theta(x_{t-1} \mid x_t)}{q(x_t \mid x_{t-1})} \right] \\
%     & =: \mathcal{L}
% \end{align}


\begin{align}
    - \mathbb{E}_{q(\mathbf{x}_0)} \left[ \log p_\theta(\mathbf{x}_0) \right]  
     &\leq  - \mathbb{E}_{q(\mathbf{x}_0)} \left[ \mathbb{E}_{q(\mathbf{x}_1,\dots,\mathbf{x}_T \mid \mathbf{x}_0)} \left[ \log \left( p(\mathbf{x}_T) \prod_{t=1}^T \frac{p_\theta(\mathbf{x}_{t-1} \mid \mathbf{x}_t)}{q(\mathbf{x}_t \mid \mathbf{x}_{t-1})} \right) \right] \right] \\
    & =  - \mathbb{E}_{q(\mathbf{x}_0,\dots,\mathbf{x}_T)} \left[ \log \left( p(\mathbf{x}_T) \prod_{t=1}^T \frac{p_\theta(\mathbf{x}_{t-1} \mid \mathbf{x}_t)}{q(\mathbf{x}_t \mid \mathbf{x}_{t-1})} \right) \right] \\
    & =   \mathbb{E}_{q(\mathbf{x}_0,\dots,\mathbf{x}_T)} \left[ -\log p(\mathbf{x}_T) - \sum_{t=1}^T \log \frac{p_\theta(\mathbf{x}_{t-1} \mid \mathbf{x}_t)}{q(\mathbf{x}_t \mid \mathbf{x}_{t-1})} \right] \\
    & =: \mathcal{L}_{\text{DDPM}}
\end{align}

one obtains the loss for training a diffusion model. This can be further transformed to

% \begin{align}
%     \mathcal{L} &= \mathbb{E}_{q(x_0,\dots,x_T)} \left[ -\log  p(x_T) - \sum_{t=2}^T  \log  \frac{p_\theta(x_{t-1} \mid x_t)}{q(x_t \mid x_{t-1})}  - \log   \frac{p_\theta(x_{0} \mid x_1)}{q(x_1 \mid x_{0})}  \right] \\
%     &= \mathbb{E}_{q(x_0,\dots,x_T)} \left[ -\log  p(x_T) - \sum_{t=2}^T  \log  \frac{p_\theta(x_{t-1} \mid x_t)}{q(x_{t-1} \mid x_t,x_0)} \cdot \frac{q(x_{t-1} \mid x_0)}{q(x_t \mid x_{0})} - \log   \frac{p_\theta(x_{0} \mid x_1)}{q(x_1 \mid x_{0})}  \right] \\
%     &= \mathbb{E}_{q(x_0,\dots,x_T)} \left[ -\log  \frac{p(x_T)}{q(x_T \mid x_0)} - \sum_{t=2}^T  \log  \frac{p_\theta(x_{t-1} \mid x_t)}{q(x_{t-1} \mid x_t,x_0)}  - \log p_\theta(x_{0} \mid x_1)  \right] \\
%     &= \mathbb{E}_{q(x_0,\dots,x_T)} \left[ \text{KL} \left[ q(x_T \mid x_0) \,\|\,  p(x_T)  \right]  + \sum_{t=2}^T  \text{KL} \left[ q(x_{t-1} \mid x_t, x_0) \,\|\,  p_\theta(x_{t-1} \mid x_t) - \log p_\theta(x_0 \mid x_1) \right]     \right]
% \end{align}


\begin{align}
\mathcal{L}_{\text{DDPM}} 
&= \mathbb{E}_{q(\mathbf{x}_0,\dots,\mathbf{x}_T)} \left[ 
    -\log p(\mathbf{x}_T) 
    - \sum_{t=2}^T \log \frac{p_\theta(\mathbf{x}_{t-1} \mid \mathbf{x}_t)}{q(\mathbf{x}_t \mid \mathbf{x}_{t-1})} 
    - \log \frac{p_\theta(\mathbf{x}_0 \mid \mathbf{x}_1)}{q(\mathbf{x}_1 \mid \mathbf{x}_0)} 
\right] \\
&= \mathbb{E}_{q(\mathbf{x}_0,\dots,\mathbf{x}_T)} \left[ 
    -\log p(\mathbf{x}_T) 
    - \sum_{t=2}^T \log \left( 
        \frac{p_\theta(\mathbf{x}_{t-1} \mid \mathbf{x}_t)}{q(\mathbf{x}_{t-1} \mid \mathbf{x}_t, \mathbf{x}_0)} 
        \cdot 
        \frac{q(\mathbf{x}_{t-1} \mid \mathbf{x}_0)}{q(\mathbf{x}_t \mid \mathbf{x}_0)} 
    \right)
    \right. \notag \\
&\quad \left.
    - \log \frac{p_\theta(\mathbf{x}_0 \mid \mathbf{x}_1)}{q(\mathbf{x}_1 \mid \mathbf{x}_0)} 
\right] \\
&= \mathbb{E}_{q(\mathbf{x}_0,\dots,\mathbf{x}_T)} \left[ 
    -\log \frac{p(\mathbf{x}_T)}{q(\mathbf{x}_T \mid \mathbf{x}_0)} 
    - \sum_{t=2}^T \log \frac{p_\theta(\mathbf{x}_{t-1} \mid \mathbf{x}_t)}{q(\mathbf{x}_{t-1} \mid \mathbf{x}_t, \mathbf{x}_0)}  - \log p_\theta(\mathbf{x}_0 \mid \mathbf{x}_1) 
\right] \\
&= \mathbb{E}_{q(\mathbf{x}_0,\dots,\mathbf{x}_T)} \left[ 
    \text{KL} \left( q(\mathbf{x}_T \mid \mathbf{x}_0) \,\|\, p(\mathbf{x}_T) \right) 
    + \sum_{t=2}^T \text{KL} \left( q(\mathbf{x}_{t-1} \mid \mathbf{x}_t, \mathbf{x}_0) \,\|\, p_\theta(\mathbf{x}_{t-1} \mid \mathbf{x}_t) \right) 
\right. \notag \\
&\quad \left. 
    - \log p_\theta(\mathbf{x}_0 \mid \mathbf{x}_1) 
\right] \quad.
\end{align}


For training, only non-constant terms are relevant. That is why the loss further simplifies to
% \begin{align}
%     \mathcal{L} &= \sum_{t=2}^T  \mathbb{E}_{q(x_0,x_t)} \left[ \text{KL} \left[ q(x_{t-1} \mid x_t, x_0) \,\|\,  p_\theta(x_{t-1} \mid x_t) \right] \right] -  \mathbb{E}_{q(x_0,\dots,x_T)} \left[\log p_\theta(x_0 \mid x_1) \right] + \text{const} \\
%     & \approx \sum_{t=2}^T  \mathbb{E}_{q(x_0,x_t)} \left[ \text{KL} \left[ q(x_{t-1} \mid x_t, x_0) \,\|\,  p_\theta(x_{t-1} \mid x_t) \right] \right] + \text{const}
%     \label{FinalDiffLoss}
% \end{align}

\begin{align}
    \mathcal{L}_{\text{DDPM}}  &= \sum_{t=2}^T  \mathbb{E}_{q(\mathbf{x}_0,\mathbf{x}_t)} \left[ \text{KL} \left( q(\mathbf{x}_{t-1} \mid \mathbf{x}_t, \mathbf{x}_0) \,\|\,  p_\theta(\mathbf{x}_{t-1} \mid \mathbf{x}_t) \right) \right] \notag \\
    &\quad -  \mathbb{E}_{q(\mathbf{x}_0,\dots,\mathbf{x}_T)} \left[\log p_\theta(\mathbf{x}_0 \mid \mathbf{x}_1) \right] + \text{const} \\
    & \approx \sum_{t=2}^T  \mathbb{E}_{q(\mathbf{x}_0,\mathbf{x}_t)} \left[ \text{KL} \left( q(\mathbf{x}_{t-1} \mid \mathbf{x}_t, \mathbf{x}_0) \,\|\,  p_\theta(\mathbf{x}_{t-1} \mid \mathbf{x}_t) \right) \right] + \text{const}
    \label{FinalDiffLoss}
\end{align}


assuming that the term $\mathbb{E}_{q(\mathbf{x}_0,\dots,\mathbf{x}_T)} \left[\log p_\theta(\mathbf{x}_0 \mid \mathbf{x}_1) \right]$ is negligible. Note that in eq. \ref{FinalDiffLoss} only Gaussians, $p_\theta(\mathbf{x}_{t-1} \mid \mathbf{x}_t) = \mathcal{N}\left( \mathbf{x}_{t-1}; \mu_\theta(\mathbf{x}_t,t),\Sigma_\theta(\mathbf{x}_t,t) \right)$  with $\Sigma_\theta(\mathbf{x}_t,t) = \sigma_t^2 \mathbf{I}$ and $q(\mathbf{x}_{t-1} \mid \mathbf{x}_t, \mathbf{x}_0) = \mathcal{N} \left( \mathbf{x}_{t-1}; \hat{\mu}(\mathbf{x}_t,\mathbf{x}_0), \hat{\beta}_t \mathbf{I}  \right)$ are compared in the \gls{KL}-divergence meaning there is a closed-form solution. The loss for the diffusion model boils down to 
\begin{equation}
    \mathcal{L}_{\text{DDPM}} = \sum_{t=2}^T \mathbb{E}_{q(\mathbf{x}_0,\mathbf{x}_t)} \left[ \frac{1}{2\sigma^2} || \hat{\mu}(\mathbf{x}_t, \mathbf{x}_0) - \mu_\theta(\mathbf{x}_t,t) ||^2  \right] \quad.
    \label{MeanLossDiffModel}
\end{equation}
The mean and the variance of the forward process $\hat{\mu}$ is obtained by 
\begin{equation}
    \hat{\mu}(\mathbf{x}_t, \mathbf{x}_0) := \frac{\sqrt{\bar{\alpha}_{t-1}} \beta_t}{1 - \bar{\alpha}_t} \mathbf{x}_0 + \frac{\sqrt{\alpha_t}(1 - \bar{\alpha}_{t-1})}{1 - \bar{\alpha}_t} \mathbf{x}_t
\end{equation}

\begin{equation}
    \hat{\beta}_t := \frac{1 - \bar{\alpha}_{t-1}}{1 - \bar{\alpha}_t} \beta_t \quad.
\end{equation}
However, instead of predicting the mean, diffusion models are more often trained on predicting the noise $\boldsymbol{\epsilon}_\theta$  because \cite{DDPM} empirically found that this lead to more stable training 

\begin{equation}
    \mathcal{L}_{\text{DDPM}}  = \sum_{t=2}^T\frac{1}{2\tilde{\beta}_t} \frac{(1 - \alpha_t)^2}{\alpha_t(1 - \bar{\alpha}_t)} \cdot \mathbb{E}_{q(x_t|x_0)} \left[ \|\boldsymbol{\epsilon}_\theta(x_t, t) - \boldsymbol{\epsilon}_t\|_2^2 \right] 
    \label{NoiseLossPreFactor}
\end{equation}
which is achieved by reparameterization of eq. \ref{MeanLossDiffModel} via 
\begin{equation}
    \hat{\mu}(\mathbf{x}_t, \mathbf{x}_0) = \frac{1}{\sqrt{\alpha_t}} \left( \mathbf{x}_t - \frac{1-\alpha_t}{\sqrt{1-\hat{\alpha_t}}}\boldsymbol{\epsilon}_t  \right) \quad.
\end{equation}


In addition, they found empirically that ignoring the weighting term of eq. \ref{NoiseLossPreFactor} worked better
\begin{equation}
    \mathcal{L}_{\text{DDPM}}  = \sum_{t=2}^T \mathbb{E}_{q(\mathbf{x}_t|\mathbf{x}_0)} \left[ \|\boldsymbol{\epsilon}_\theta(\mathbf{x}_t, t) - \boldsymbol{\epsilon}_t\|_2^2 \right] 
    \label{NoiseLoss}
\end{equation}


\subsubsection{DDPM Sampler}  Algorithm \ref{AlogDDPMTraining} and fig. \ref{FigDDPMTraining} depict the training process of the DDPM approach. In each training iteration, an image $\mathbf{x}_0 \sim p_\text{data}$ is sampled from the data distribution. Next, a timestep $t$ is uniformly sampled from a predefined interval $[0,...,T]$ and Gaussian noise is drawn according to $\boldsymbol{\epsilon} \sim \mathcal{N}(\mathbf{0}, \mathbf{I})$. Using eq. \ref{x_t} the noisy image $\mathbf{x}_t$ at timestep $t$ is computed. The diffusion model is applied to predict the noise from this noisy input, and the loss function (eq.  \ref{NoiseLoss}) measures the difference between the predicted and true noise. Finally, the model parameters are updated using gradient descent or a more advanced iterative optimization algorithms. \\



\begin{algorithm}
\caption{Training of \gls{DDPM} using learning rate $\eta$ \cite{LectureNotesDDPM}}
\begin{algorithmic}
\Repeat
\State $\mathbf{x}_0 \sim p_{\text{data}}$
\State $t \sim \text{Uniform}(\{1, \ldots, T\})$
\State $\boldsymbol{\epsilon} \sim \mathcal{N}(\mathbf{0}, \mathbf{I})$
\State $\mathbf{x}_t \leftarrow \sqrt{\bar{\alpha}_t} \cdot \mathbf{x}_0 + \sqrt{1 - \bar{\alpha}_t} \cdot \boldsymbol{\epsilon}$
\State $\boldsymbol{\theta} \leftarrow \boldsymbol{\theta} - \eta \cdot \nabla_{\boldsymbol{\theta}} \|\boldsymbol{\epsilon}_{\boldsymbol{\theta}}(\mathbf{x}_t, t) - \boldsymbol{\epsilon}\|_2^2$
\Until{converged}
\end{algorithmic}
\label{AlogDDPMTraining}
\end{algorithm}

\begin{figure}[H]
    \centering
    \includegraphics[width=0.8\textwidth]{images/Background/DDPM_Training.png}  % adjust filename and width
    \caption{DDPM training  \cite{DDPMSamplerImage}.}
    \label{FigDDPMTraining}
\end{figure}


Algorithm \ref{AlogDDPMSampling} and fig. \ref{FigDDPMSampling} show schematically the sampling process which reverses the diffusion process to generate images from noise. Starting with pure Gaussian noise $\mathbf{x}_T \sim \mathcal{N}(\mathbf{0}, \mathbf{I})$, the process iteratively denoises the image. At each timestep $t$, the slightly denoised image $\mathbf{x}_{t-1}$ is computed using
\begin{equation}
    \mathbf{x}_{t-1} = \frac{1}{\sqrt{\alpha_t}} \left(  \mathbf{x}_t - \frac{1-\alpha_t}{\sqrt{1-\bar{\alpha}_t}} \boldsymbol{\epsilon}_\theta(\mathbf{x}_t,t)  \right) + \sigma_t \mathbf{z}
\end{equation}
until $t=2$. In the last step, the final clear images is obtained by 
\begin{equation}
    \mathbf{x}_{0} = \frac{1}{\sqrt{\alpha_1}} \left(  \mathbf{x}_1 - \frac{1-\alpha_1}{\sqrt{1-\bar{\alpha}_1}} \boldsymbol{\epsilon}_\theta(\mathbf{x}_1,1)  \right) + \sigma_1 \mathbf{z}
\end{equation}
Here, it becomes clear that the model has to predict the noise $T$ times which makes inference of such a diffusion model slow.





\begin{algorithm}
\caption{Sampling of \gls{DDPM} \cite{LectureNotesDDPM}}
\begin{algorithmic}

\State $\mathbf{x}_T \sim \mathcal{N}(\mathbf{0}, \mathbf{I})$
\For{$t = T, \dots, 1$}
    \If{$t > 1$}
        \State $\lambda \gets 1$
    \Else
        \State $\lambda \gets 0$
    \EndIf
    \State $\mathbf{z} \sim \mathcal{N}(\mathbf{0}, \mathbf{I})$
    \State $\mathbf{x}_{t-1} \gets \frac{1}{\sqrt{\alpha_t}} \left( \mathbf{x}_t - \frac{1 - \alpha_t}{\sqrt{1 - \bar{\alpha}_t}} \cdot \boldsymbol{\epsilon}_\theta(\mathbf{x}_t, t) \right) + \lambda \cdot \sigma_t \cdot \mathbf{z}$
\EndFor
\State \Return $\mathbf{x}_0$

\end{algorithmic}
\label{AlogDDPMSampling}
\end{algorithm}


\begin{figure}[H]
    \centering
    \includegraphics[width=1.\textwidth]{images/Background/DDPM.png}  % adjust filename and width
    \caption{DDPM sampling \cite{DDPMSamplerImage}.}
    \label{FigDDPMSampling}
\end{figure}

\subsection{Denoising Diffusion Implict Models}
The \gls{DDPM} sampling process can be accelerated by replacing the Markovian diffusion process with a non-Markovian alternative resulting in denoising diffusion implicit models (\gls{DDIM}s) \cite{DDIM}. Unlike \gls{DDPM}, \gls{DDIM} defines the joint distribution as 
\begin{equation}
    q(\mathbf{x}_{1:T} \mid \mathbf{x}_0) = q(\mathbf{x}_T  \mid \mathbf{x}_0) \prod_{t=2}^T q(\mathbf{x}_{t-1} \mid \mathbf{x}_t, \mathbf{x}_0) \quad.
\end{equation}

It becomes clear that the forward process $q(\mathbf{x}_t \mid \mathbf{x}_{t-1}, \mathbf{x}_0)$ is not Markovian anymore due to its explicit dependence on $\mathbf{x}_0$. The authors of \cite{DDIM} parameterize the reverse process as

\begin{equation}
q(\mathbf{x}_{t-1} \mid \mathbf{x}_t, \mathbf{x}_0) = 
\mathcal{N} \left( 
\mathbf{x}_{t-1}; 
\sqrt{\bar{\alpha}_{t-1}} \cdot \mathbf{x}_0 + 
\sqrt{1 - \bar{\alpha}_{t-1} - \tilde{\sigma}_t^2} \cdot 
\frac{\mathbf{x}_t - \sqrt{\bar{\alpha}_t} \mathbf{x}_0}{\sqrt{1 - \bar{\alpha}_t}}, 
\tilde{\sigma}_t^2 \cdot \mathbf{I}
\right) \quad.
\end{equation}
The corresponding sampling equation becomes
\begin{equation}
\mathbf{x}_{t-1} = \underbrace{\frac{1}{\sqrt{\alpha_t}} \left(\mathbf{x}_t - \sqrt{1 - \bar{\alpha}_t} \cdot \boldsymbol{\epsilon}_\theta(\mathbf{x}_t, t)\right)}_{\text{Predicted } \mathbf{x}_0} + \underbrace{\sqrt{1 - \bar{\alpha}_{t-1} - \tilde{\sigma}_t^2} \cdot \boldsymbol{\epsilon}_\theta(\mathbf{x}_t, t)}_{\text{Direction of } \mathbf{x}_t} + \underbrace{\tilde{\sigma}_t \cdot \mathbf{z}}_{\text{Noise}}
\label{DDIM_Sampler}
\end{equation}
where $\tilde{\sigma}_t$ is a parameter to choose. $\mathbf{x}_{t-1}$ consists of three parts. The first part is a prediction of $\mathbf{x}_0$ based on $\boldsymbol{\epsilon}_\theta$. The second part is a directional term describing the transition between $\mathbf{x}_t$ and $\mathbf{x}_0$. And the last part adds Gaussian noise scaled by $\tilde{\sigma}_t$. 
The key advantage of \gls{DDIM} is the flexibility in choosing $\tilde{\sigma}_t$, which does not need to follow a predefined variance schedule. Different values of $\tilde{\sigma}_t$ yield different diffusion processes while using the same trained model. In particular, for the choice 
\begin{equation}
\sigma_t = \sqrt{\frac{1 - \bar{\alpha}_{t-1}}{1 - \bar{\alpha}_t}} \sqrt{1 - \frac{\bar{\alpha}_t}{\bar{\alpha}_{t-1}}}
\end{equation}
the \gls{DDIM} approach becomes Markovian again and here, the deep connection between \gls{DDPM} and \gls{DDIM} can be seen. Another aspect to note is that if setting $\tilde{\sigma} = 0$ the whole process becomes deterministic.


\subsection{Scored-Based Generative Models}
This disscussion is mainly based on \cite{Survey2, SurveyDM}.
Score-Based Generative Models (\gls{SGM}s) \cite{IntroSGM, ImprovedSGMs} do not learn directly the data distribution $p_\text{data}(\mathbf{x})$ but the score of the probability function $\nabla_x \log p_\text{data}(\mathbf{x})$. The score represents a vector field that points to the steepest increase in log probability density, thereby, guiding the data generation process to regions of higher data density. There are several approaches on how the score function can be learned \cite{ScoreBasedODE, SteinDiscrepancy, SlicedScoreMatching}. Here, the one closest to the \gls{DDPM} approach is presented \cite{IntroSGM}. During training, the data is perturbed by Gaussian noise at different noise levels and a neural network is trained to estimate the score function based on the noise level $s_\theta(\mathbf{x}_t, t) \approx \nabla_{\mathbf{x}_t} \log q(x_t)$. These networks are called Noise-Conditional Score Networks (\gls{NCSN}s). The learning objective boils down to 

\begin{align}
   &\mathbb{E}_{T \sim \mathcal{U}[1,T], \mathbf{x}_0 \sim q(\mathbf{x}_0), \mathbf{x}_t \sim q(\mathbf{x}_t|\mathbf{x}_0)} \left[ \lambda(t) \sigma_t^2 \left\| \nabla_{\mathbf{x}_t} \log q(\mathbf{x}_t) - s_\theta(\mathbf{x}_t, t) \right\|^2 \right] 
   \label{SGMObjectiveFunction}\\
 &= \mathbb{E}_{T \sim \mathcal{U}[1,T], \mathbf{x}_0 \sim q(\mathbf{x}_0), \mathbf{x}_t \sim q(\mathbf{x}_t|\mathbf{x}_0)} \left[ \lambda(t) \sigma_t^2 \left\| \nabla_{\mathbf{x}_t} \log q(\mathbf{x}_t | \mathbf{x}_0) - s_\theta(\mathbf{x}_t, t) \right\|^2 \right] + \text{const} \\
 &= \mathbb{E}_{T \sim \mathcal{U}[1,T], \mathbf{x}_0 \sim q(\mathbf{x}_0), \mathbf{x}_t \sim q(\mathbf{x}_t|\mathbf{x}_0)} \left[ \lambda(t) \left\| -\frac{\mathbf{x}_t - \mathbf{x}_0}{\sigma_t} - \sigma_t s_\theta(\mathbf{x}_t, t) \right\|^2 \right] + \text{const} \\
 &= \mathbb{E}_{T \sim \mathcal{U}[1,T], \mathbf{x}_0 \sim q(\mathbf{x}_0), \boldsymbol{\epsilon} \sim \mathcal{N}(\mathbf{0}, \mathbf{I})} \left[ \lambda(t) \left\| \boldsymbol{\epsilon} + \sigma_t s_\theta(\mathbf{x}_t, t) \right\|^2 \right] + \text{const}
 \label{SGMLoss}
\end{align}
In the final line in eq. \ref{SGMLoss} the relation to \gls{DDPM} loss becomes visible by identifying $\boldsymbol{\epsilon}(\mathbf{x}_t,t) = - \sigma_t s_\theta(\mathbf{x}_t,t)$ \cite{Survey2}, showing that learning to predict noise is equivalent to learning the score function. \\
For sampling a new element from the data distribution \cite{IntroSGM} proposed Annealed Langevin Dynamics (\gls{ALD}) which produces samples by only using the score function being approximated by the learned neural network
\begin{equation}
    \mathbf{x}_t = \mathbf{x}_{t-1} + \alpha \nabla_{\mathbf{x}} \log p(\mathbf{x}_{t-1}) + \sqrt{2\alpha} \mathbf{z}_t, \quad 1 \leq t \leq T
    \label{SamplingSGM}
\end{equation}
with $\mathbf{z}_t \sim \mathcal{N}(\mathbf{0}, \mathbf{I})$ and $\alpha$ regulates the step size of the update in the direction of the score. By applying eq. \ref{SamplingSGM} iteratively to initial Gaussian noise, a new element of the data distribution is generated.


\subsection{Stochastic Differential Equations}
This discussion is mainly based on \cite{Survey2, SurveyDM}.
Stochastic Differential Equations (\gls{SDE}s) \cite{ScoreBasedODE} are a generalization of \gls{SGM}s and \gls{DDPM}s to continuous time where the forward and the reverse process are modeled by the solution of stochastic differential equations. The forward process is described by the \gls{SDE}
\begin{equation}
    d\mathbf{x} = \mathbf{f}(\mathbf{x},t) dt + g(t) d\mathbf{w}
    \label{SDE}
\end{equation}
where $\mathbf{f}(\mathbf{x},t)$ is the drift coefficient, $g(t)$ the diffusion coefficient and $\mathbf{w}$ a standard Wiener process (aka Brownian Motion). \cite{Anderson} proved that the diffusion process in eq. \ref{SDE} can be reversed leading to the reversed-time \gls{SDE} given by 
\begin{equation}
    d\mathbf{x} = \left[ \mathbf{f}(\mathbf{x},t) - \frac{1}{2} g(t)^2 \nabla_x \log q(\mathbf{x}_t) \right] dt + g(t) d\mathbf{\tilde{w}}
    \label{reverseSDE}
\end{equation}
with a standard Wiener process $\mathbf{\tilde{w}}$ where the time flies backward. Both, forward and reverse \gls{SDE}, share the same marginals. A score neural network $s_\theta$ is trained similarly to \gls{SGM}s by generalizing the objective function (see eq. \ref{SGMObjectiveFunction}) to continuous time 
\begin{equation}
    \mathbb{E}_{t \sim \mathcal{U}[0,T], \mathbf{x}_0 \sim q(\mathbf{x}_0), \mathbf{x}_t \sim q(\mathbf{x}_t|\mathbf{x}_0)} \left[ \lambda(t) \left\| s_\theta(\mathbf{x}_t, t) - \nabla_{\mathbf{x}_t} \log q_{t}(\mathbf{x}_t) \right\|^2 \right]
\end{equation}
such as $s_\theta(\mathbf{x},t) \approx \nabla_{\mathbf{x}_t} \log q_{t}(\mathbf{x}_t) $. For converting noise to data, a solution of eq. \ref{reverseSDE} has to be computed using the score neural network and a \gls{SDE} solver. Another finding of \cite{ScoreBasedODE} is that for eq. \ref{reverseSDE} an \gls{ODE} exists having the same marginal as the \gls{SDE} being known as probability flow \gls{ODE} in the literature. It's given by by the deterministic version of eq. \ref{reverseSDE}
\begin{equation}
    d\mathbf{x} = \left[ \mathbf{f}(\mathbf{x},t) - \frac{1}{2} g(t)^2 \nabla_x \log q(x_t) \right] dt
\end{equation}
Both reversed-\gls{SDE} or probability flow \gls{ODE} can be used to generate new samples using \gls{SDE} solvers \cite{ScoreBasedODE, FastSDE} or respectively \gls{ODE} solvers \cite{ScoreBasedODE, ODESolver1, ODESolver2, ODESolver3}.




\subsection{Flow Matching}
The concept of flow matching was introduced for normalizing flows \cite{FlowMatching} to find transport trajectories between two probability distributions. In contrast to diffusion models, which follows a noisy trajectory flow matching directly learns a probability path $p(t,x)$ connecting two distributions $\pi_1$ and $\pi_2$. In particular, the corresponding vector field $u(t,x)$, often referred to as velocity, that generates the probability path is learned via the flow matching loss function 
\begin{equation}
    \mathcal{L}_{\text{FM}} = \mathbb{E}_{t \sim \mathcal{U}(0,1=), x \sim p(t,x)} \left[ \left\| v_\theta(t,x) - u(t,x) \right\|^2 \right]  
    \label{LFM}
\end{equation}
by a neural network $v_\theta(t,x)$. The physical analogy is that particles are transport from one location to another and the vector field specifies at every time and location in which direction and how fast the particles move, in order to reach a final state. However, in flow matching neither the probability path $p(t,x)$  nor the vector field $u(t,x)$ is directly accessible. Therefore, the marginal probability path is constructed by a mixture of simpler conditional probability paths $p(t,x \mid x_1)$ such as $p(0,x, \mid x_1) = p(0,x) \approx q(x)$ and $p(1,x \mid x_1) = \mathcal{N}(x; \mathbf{0}, \mathbf{I})$ with $x_1$ being sampled from the data distribution
\begin{equation}
    p(x,t) = \int p(t,x \mid x_1) q(x_1) dx_1
\end{equation}
where $q(x)$ denotes the data distribution.
Furthermore, a marginal vector field can be formulated 
\begin{equation}
    u(t,x) = \int u(t,x \mid x_1) \frac{p(t,x \mid x_1) q(x_1)}{p(t,x)} dx_1
\end{equation}
based on the conditional vector field $u(t,x \mid x_1)$ which generates the marginal probability path. The trick is not to use the marginal distributions, because the integrals are still intractable, instead reformulate the loss function (eq. \ref{LFM}) using the conditional vector field. 
\begin{equation}
    \mathcal{L}_{\text{CFM}}(\theta) = \mathbb{E}_{t \sim \mathcal{U}(0,1=), x \sim p(t,x)} \left[ \left\| v_\theta(t,x) - u(t,x \mid x_1) \right\|^2 \right]  
    \label{LCFM}
\end{equation}
The key insight is that $\mathcal{L}_{\text{CFM}}$ leads to the same gradients w.r.t $\theta$ as $\mathcal{L}_{\text{FM}}$ and can actually be computed due to the tractable conditional vector field.

\subsubsection{Rectified Flow}
\cite{RectifiedFlowMatching} proposed rectified flow, a specialized variant of flow matching \cite{FlowMatching} that aims to straighten the trajectories between source and target distributions $\pi_0$ and $\pi_1$
by iteratively refining the learned \gls{ODE}. This approach offers significant computational advantages, specifically, the straightened paths create a more direct mapping between distributions, making the velocity field easier to learn and enabling efficient sampling with as few as 1-2 Euler integration steps, compared to the 50-1000 steps typically required by diffusion models. \\ The \gls{ODE} of the rectified flow for $x_0 \sim \pi_0$ and $x_1 \sim \pi_1$ is given by 
\begin{equation}
dx_t = v_\theta(x_t,t)dt
\label{ODERectifiedFlow}
\end{equation}
where $v_\theta$ is the vector field approximated by a neural network driving the flow such that it follows the direction $x_1-x_0$. This is enforced by the learning objective 
\begin{equation}
    \min_v \int_0^1 \mathbb{E} \left[ \left\| (x_1 - x_0) - v_\theta(x_t, t) \right\|^2 \right] dt
\end{equation}
where $x_t = tx_1 + (1-t)x_0$ is the linear interpolation between $x_0$ and $x_1$. After training, the \gls{ODE} in eq. \ref{ODERectifiedFlow} can be solved by sampling $x_0$, thus obtaining a sample from the data distribution $\pi_1$. One main property of the flow obtained through this learning procedure is that trajectories of the rectified flow do not cross. This is depicted in fig. \ref{RectifiedFlow}. (a) shows the linear interpolations from pairs of $(x_0, x_1) \sim \pi_0 \times \pi_1$. In this case, trajectories intersect. Next, in (b), the vector field is learned. The rectified flow induced by $(x_0, x_1)$ results in rewired trajectories such that they do not cross. This procedure, obtaining the linear interpolation between the elements from the previously learned rectified flow $x^{k-1}_0, x^{k-1}_1$ and learning the rectified flow $x_t^k$, can be iteratively applied and is called Reflow. The second iteration is depicted in (c) and (d). With every iteration the trajectory becomes more straightened (see algorithm \ref{RectifieFlowAlog}). An optional step to enable fast inference is to learn the $k$-rectified flow by a neural network $\hat{T}$ which directly maps $x_0^k$ to $x_1^k$. Note, the difference to rectified flow learning is that here, the network approximates the mapping between $(x^k_0,x^k_1)$ whereas rectification yields a different mapping with lower transport cost and straightened trajectory $(x_0^{k+1}, x_1^{k+1})$.


\begin{figure}[H]
    \centering
    \includegraphics[width=1.\textwidth]{images/Background/Overview_2.png}  % adjust filename and width
    \caption{Rectified Flow \cite{RectifiedFlowMatching}.}
    \label{RectifiedFlow}
\end{figure}

\begin{algorithm}[H]
\caption{Rectified Flow: Main Algorithm}
\begin{algorithmic}
\State \textbf{Procedure}: $x' = \text{RectFlow}((x_0, x_1))$
\State \textbf{Inputs:} Draws from a coupling $(x_0, x_1)$ of $\pi_0$ and $\pi_1$; velocity model $v_\theta : \mathbb{R}^d \to \mathbb{R}^d$ with parameter $\theta$.
\State \textbf{Training:} $\hat{\theta} = \arg \min_\theta \mathbb{E} \left[ \left\| x_1 - x_0 - v(tx_1 + (1-t)x_0, t) \right\|^2 \right]$, with $t \sim \text{Uniform}([0,1])$.
\State \textbf{Sampling:} Draw $(x_0', x_1')$ following $dx_t = v_{\hat{\theta}}(x_t, t)dt$ starting from $x'_0 \sim \pi_0$ (or backwardly $x'_1 \sim \pi_1$).
\State \textbf{Return:} $Z = \{Z_t : t \in [0,1]\}$.

\State
\State \textbf{Reflow (optional):} $x^{k+1} = \text{RectFlow}((x_0^k, x_1^k))$, starting from $(x_0^0, x_1^0) = (x_0, x_1)$. 
\State \textbf{Distill} (optional): Learn a neural network $\hat{T}$ to distill the $k$-rectified flow, such that $x^k_1 \approx \hat{T}(x^k_0)$ .
\label{RectifieFlowAlog}
\end{algorithmic}
\end{algorithm}
